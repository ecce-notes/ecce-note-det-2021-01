The purpose of this note is to document the plans by the ECCE consortium to re-use the BaBar solenoid.  The BaBar solenoid is currently planned for use in the sPHENIX experiment and will be available after sPHENIX running concludes in 2025. The solenoid provides a 1.4T central field at design current. 

The design parameters of the BaBar solenoid are shown in Table~\ref{tab:BaBarStats}. 

%begin{figure}[h!tbp]
%   \centering
%    \includegraphics[width=0.9\textwidth]{figs/BaBar_Stats.png}
%    \caption{Parameters of the BaBar solenoid.}
%    \label{fig:BaBarStats}
%\end{figure}

\begin{table}[htb]
\footnotesize
\begin{tabularx}{\linewidth}{*{2}{>{\centering\arraybackslash}X}}
\begin{tabular}[b]{|l|l|}
\hline
        Central Induction &  1.5T \\ \hline
        Conductor Peak Field &  2.3T \\ 
        Winding structure &  2 layers \\ 
                          &  graded current density \\ 
        Uniformity in tracking region &  $\pm3\%$ \\ 
        Winding Length &  3512 mm \it{at R.T.} \\ 
        Winding mean radius &  1530 mm \it{at R.T.} \\ 
        Operating Current &  4596 A (4650 A$^{*}$)\\ 
        Inductance &  2.57 H (2.56 H$^{*}$)\\  
        Stored Energy &  27 MJ\\ 
        Total Turns &  1067 \\ 
        Total Length of Conductor &  10,300 m \\ \hline
        $^{*}$ Design Value &   \\ \hline
\end{tabular}
&
\includegraphics[scale=1.3]{figs/BaBar_Shipping.png}     \\
\caption{Main characteristics of the BaBar solenoid, as built.}
\label{tab:BaBarStats}
&
\captionof{figure}{The BaBar solenoid in its shipping cradle.} 
\label{fig:BaBarStats}
\end{tabularx}
\end{table}

\begin{figure}[h!tbp]
    \centering
    \includegraphics[width=0.9\textwidth]{figs/magnet_install_1.png}
    \caption{The BaBar solenoid in October, 2021, as it was installed in the sPHENIX experiment. The solenoid is resting in the barrel flux return, which will be completed with additional sectors in the coming months.  The experimental cradle, barrel flux return (outer hadronic calorimeter), and BaBar solenoid are all items planned to be re-used by the ECCE experiment.}
    \label{fig:BaBarInSPHENIX}
\end{figure}


% 
% Can be dropped as only one subsection
%
%\subsection{Refurbishment of the BaBar Solenoid}

The re-use of the magnet has been the subject of an engineering study and risk analysis, available as an EIC Technical Note (EICTJ-O-DE-PLT-TD-0017-R00)~\cite{BaBarAnalysis}, which also details the potential actions required to refurbish the BaBar solenoid for use in ECCE. 

After extensive discussions with the JLab engineers, it was decided that any modifications or refurbishment that required opening the BaBar solenoid cryostat would not be worth the additional risk. None of these actions will be necessary if the magnet continues to operate well throughout sPHENIX running. sPHENIX is expected to conduct a high-field magnet test in the experiment flux return (which will also be re-used for ECCE) in mid-2022, followed by experimental operations 2023-25. Figure~\ref{fig:BaBarInSPHENIX} shows the BaBar solenoid installed in the sPHENIX experiment. As a mitigation against the schedule risk posed by a problem with the BaBar solenoid developing late in sPHENIX running, we proceed with the initial engineering and design for a replacement magnet. It is expected a final decision to proceed with the BaBar solenoid or produce a new magnet will be taken in mid-2023 after the performance of the BaBar solenoid during the first year of sPHENIX running is reviewed by a panel of experts.  The risk-mitigation decision tree is shown in Figure~\ref{fig:risk_tree}. 

A draft schedule for the ECCE solenoid is shown in Figure~\ref{fig:magnet_schedule}. In Q4FY23, the ECCE collaboration will convene a panel of experts to review the operation of the BaBar solenoid in sPHENIX as well as the design for a replacement magnet. Prior to this review, we will send out an RFI to potential manufacturers so that information is available for the review.  Based on the information available at this time, a decision will be made to either proceed with reuse of the BaBar solenoid or the procurement of a replacement magnet. 

\begin{figure}[h!tbp]
    \centering
    \includegraphics[width=0.9\textwidth]{figs/flow.png}
    %\includegraphics[width=0.9\textwidth]{figs/flowchart_Page_1.png}
    \caption{
   Flowchart for the selection of either the BaBar magnet or a replacement magnet with the same characteristics.}
    \label{fig:risk_tree}
\end{figure}

\begin{figure}[h!tbp]
    \centering
    \includegraphics[width=1.0\textwidth]{figs/BaBar-Timeline.png}
    %ECCE Detector Solenoid_Babar reuse_v5.png}
    \caption{Schedule for the ECCE solenoid.}
    \label{fig:magnet_schedule}
\end{figure}